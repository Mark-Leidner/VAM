%x Emacs: -*- mode:outline; outline-regexp:"[%\\\\]x+ " -*-
%  RCS: $Id: fgat-error-analysis.tex,v 1.3 2006/09/13 22:06:24 leidner Exp $
%  Copy of RCS: Id: progress-Aug04.tex,v 1.4 2004/09/29 15:18:05 rnh Exp
%  p1219 progress report as of the end of August 2004.

\documentclass[12pt,titlepage]{article}

\input{format}
\input{abbrev}
\usepackage[dvips]{color}
\usepackage{ametsoc}
\usepackage[final]{graphicx}
\usepackage{rotate, epsf}

\graphicspath{
      {/project/dap/fig/p1326/}}

\newcommand{\authors}{Ross N. Hoffman and S. Mark Leidner}
\newcommand{\RCSid}{$Id: fgat-error-analysis.tex,v 1.3 2006/09/13 22:06:24 leidner Exp $}
\newcommand{\RCSrevision}{$Revision: 1.3 $}
\newcommand{\support}{\thanks{ 
        Supported by NASA contract NAS5-32953.
        AER, Inc. intends to retain % (has retained)
        patent rights to certain aspects of the algorithms described herein
        under FAR 52.227-11.}}
\newcommand{\docinfo}{\thanks{ 
        To be submitted to Goddard Space Flight Center (NASA/GSFC),
        Greenbelt, MD 20771.}}
\newcommand{\rightsinfo}[1]{\thanks{ 
        Copyright \copyright\ #1.  
        Work in Progress.  All Rights Reserved.}}
\newcommand{\aerinfo}{\thanks{
        Atmospheric and Environmental Research (AER), Inc.
        131 Hartwell Avenue, Lexington, MA 02421-3126.
        Phone: (781)-761-2288   FAX: (781)-761-2299
        Net: \computer{http://www.aer.com/}.}}
\newcommand{\revinfo}[1]{\thanks{
        AER document version control: \computer{P1326, #1}.
        Formatting version \computer{$Id: fgat-error-analysis.tex,v 1.3 2006/09/13 22:06:24 leidner Exp $}.}}
\newcommand{\xtitle}[4]{
        \title{Support for the Development of Cross-Calibrated, \\
               Multi-Platform Ocean Surface Wind Products
% For delivery modify \docinfo and eliminate \rightsinfo
        #1$\!\!$\docinfo} 
        \author{#2\rightsinfo{#2} \\
        Atmospheric and Environmental Research, Inc.\aerinfo }
        \date{#3\revinfo{#4} \\~\\ \today}
        \maketitle
        \markboth {AER, Inc.\ NASA FGAT for the VAM}
        {#1 (#3)}
        \begin{titlepage} \mbox{~} \end{titlepage}
}

\newcommand{\x}[1]{\xtitle{#1}{\authors}{\RCSrevision}{\RCSid}}

\begin{document}

\x {}

\xx {Introduction}

 This is a report on AER's support of
the generation of cross-calibrated, multi-platform ocean surface wind
products at NASA Goddard.
 The project at NASA Goddard aims to generate high-quality,
high-resolution, vector winds over the world's oceans between the
years 1987 and 2007 using {\em in situ} and remotely sensed wind
observations.
This report summarizes one task supported by AER project number 1326.

 AER is helping specifically by enhancing the capability and
optimizing the use of a two-dimensional data assimilation application
for surface winds called the variational analysis method (VAM).
 The VAM is at the center of the project's analysis procedures for
combining observations of the wind over the world's oceans.

In this report, the results of analysis and software development for
accounting for time interpolation errors of the First Guess at the
Appropriate Time (FGAT) procedure are detailed.

First \secr{FGAT} we repeat a description of the FGAT procedure
that we previously reported on (in our August 2004 progress report).

Second in \secr{errors} we describe the calculation of FGAT time
interpolation errors for ocean surface winds from several detailed
MM5 mesoscale model forecasts.
(These forecasts were available from another project.)
It is clear from this analysis that ocean surface wind errors---$u$, $v$,
and $V$----increase rapidly with time offset and asymptote at about
\ms{1} by 3 hours.

Finally, in \secr{Jnew} we reformulate the VAM obs functions to
include this effect.

\xx {Data assimilation using the First Guess at the Appropriate Time (FGAT) \secl{FGAT}}

The First Guess at the Appropriate Time (FGAT) procedure was designed
and coded previously.
The design is detailed below.

\xxx {FGAT design\secl{design}}

FGAT can be used to account for the first order effects of the
asynopticity of the ocean surface wind observations.
For a basic FGAT procedure one uses the time interpolated first guess
to define the observation increments, but assumes the analysis
increments are constant with respect to time.
Another approach is to interpolate in time through the analysis (\ie,
the analysis being determined by the VAM at the synoptic time) at the
center time and through the fixed earlier and later estimates of the
wind field.
As described below both cases can be written as
 \eql{fgat}{V_a(t) = \alpha V_a(t_1) + V_{\delta}.}
Here $V = (u,v)$ is the surface wind, $t$ is time, and $t_1$ is the
synoptic time of the analysis.
Subscripts $a$, $b$, and $\delta$ denote the analysis, the
background, and the FGAT increments, respectively.
Since $V_a(t)$ is only needed to simulate observations \eqr{fgat}
needs only be evaluated at the observation locations.
Therefore we have added $\alpha$, $u_{\delta}$, and $v_{\delta}$ to
the VAM observation data structure.
Preprocessors should  set  $\alpha=1$, $u_{\delta}=0$, and
$v_{\delta}=0$ so that there is no time interpolation.
\computer{FGAT\_ADD} is designed to calculate $\alpha$, $u_{\delta}$, and
$v_{\delta}$ for a variety of options governing the time
interpolation.

In what follows we determine $\alpha$ and $V_{\delta}$ for linear or
quadratic time interpolation, and for FGAT or analysis time
interpolation.
Three possible modes of operation are then described.

\xxxx {Three point time interpolation}

Suppose we are given data values $(x_0, x_1, x_2)$ at three times
$(t_0, t_1, t_2)$.
The interpolated value is linear in the data, so we may write
 \eql{three}{x(t) = c_0(t) x_0 + c_1(t) x_1 + c_2(t) x_2 .}
In the linear case one of $(c_0, c_2)$ will be zero and the
other $c_k$ will vary linearly with $t$.
In the quadratic case the $c_k$ are quadratic functions of time.

To determine the $c_k$ we use the Lagrange form of the collocating
polynomial of degree $n$ for data points  $x_0, x_1, \ldots x_n$:
 \eql{Lagrange}{x(t) = \sum_{k=0}^n x_k
     \left[\prod_{i \ne k} (t-t_i) \right]
     \left[\prod_{i \ne k} (t_k-t_i) \right]^{-1}
   = \sum_{k=0}^n c_k(t) x_k .}
Note that each $c_k$ is a degree $n$ polynomial in $t$ and that
$c_k(t_i) = \delta_{ik}$ thereby ensuring collocation.

For our problem $t_2 - t_1 = t_1 - t_0 = \delta t$.
Then our results for $c_k$ are most concise if we center and normalize
time according to 
 \eql{time}{\Delta = (t - t_1)/\delta t .}
There are three cases to consider.
For linear interpolation between $t_0$ and $t_1$ we have $(c_0 =
-\Delta, c_1 = \Delta + 1, c_2 = 0)$.
For linear interpolation between $t_1$ and $t_2$ we have $(c_0 = 0, c_1
= 1 - \Delta, c_2 = \Delta)$.
For quadratic interpolation between $t_0$ and $t_2$ we have $(c_0 =
\Delta(\Delta - 1)/2, c_1 = -(\Delta + 1)(\Delta - 1), c_2 = (\Delta +
1)\Delta/2)$.
Since $\sum c_k = 1$ we can always calculate $c_1$ as $1-(c_0 + c_2)$.

\xxxx {FGAT time interpolation}

Since the analysis increment is fixed with respect to time we may
write the analysis at any time as the sum of the background at that
time plus the analysis increment:
 \eql{fgat1}{V_a(t) = V_b(t) + [V_a(t_1) - V_b(t_1)] .}
In terms of the background values at the three time levels we have
 \eql{fgat2}{V_a(t) = V_a(t_1) + c_0(t) V_b(t_0) + (c_1(t) - 1)
     V_b(t_1) +  c_2(t) V_b(t_2) .}
Comparing this result with \eqr{fgat} we see that $\alpha = 1$ and
$V_{\delta} = c_0(t) V_b(t_0) + (c_1(t) - 1) V_b(t_1) + c_2(t)
V_b(t_2)$.

\xxxx {Analysis time interpolation}

For analysis time interpolation we simply have:
 \eql{anal}{V_a(t) = c_0(t) V_b(t_0) + c_1(t) V_a(t_1) +  c_2(t) V_b(t_2) .}
Comparing this result with \eqr{fgat} we see that $\alpha = c_1(t)$
and $V_{\delta} = c_0(t) V_b(t_0) + c_2(t) V_b(t_2)$.

\xxxx {Modes of operation}

Usually the analysis is applied sequentially.
This allows us to consider three modes of operation, different only
in how $V_b(t_0)$ and $V_b(t_2)$ are defined.
In all three modes $V_b(t_1)$ should be defined as the \apriori\
background.
In the \emph{analysis mode} $V_b(t_0)$ and $V_b(t_2)$ are both the
\apriori\ background.
In the \emph{filter mode} $V_b(t_0)$ is the result of the analysis
determined at the previous synoptic time.
In the \emph{smoother mode}, after a filter mode sequence, we filter
again, but backwards in time so that $V_b(t_0)$ is the result of the
forward filter and $V_b(t_2)$ is the result of the backwards filter.

\xxx {FGAT assessment\secl{assess}}

The VAM using FGAT was tested extensively and some anomalies were
found.

Problems typically occur when overlapping swaths (especially of wind
speed only data) are offset in time by several hours.

\xx {Analysis of FGAT errors\secl{errors}}

To quantify FGAT errors, we use mesoscale forecasts generated by the
Weather Research and Forecast (WRF) model.
 We consider hourly WRF forecasts to be truth, and then measure
the difference between linear-in-time interpolated forecast fields
($\delta t =$ 6 h) and the ``truth''.
 We compute the FGAT error as an RMS of surface wind field
differences, $(X_i - X_t)$, where $X$ is one of ($u$, $v$, $|V|$),
and subscripts $i$ and $t$ refer to the linearly interpolated 
values and truth, respectively.
 We take these differences to be a measure of the error introduced by
the linear time-interpolated FGAT described in \secr{FGAT}

We applied this idea to three different meteorological conditions
in three different geographic regions, to sample the range of
typical FGAT errors.
 The WRF forecast domains are all 201 $\times$ 201 $\times$ 31
gridpoints with a \mks{27}{km} grid spacing in the horizontal.
 The upper lid of the model is \mks{50}{hPa}.
 Four-day forecasts were generated for each region, using GFS FNL
analyses to supply initial and lateral boundary conditions.
 Only forecasts after the first 24 hours were considered to avoid
any spin-up features in the model fields.

 \figr{wind-diffs} shows sample surface wind fields from
each domain (left column) and corresponding differences (right
column) between 3 hour interpolated surface winds and the
``truth'' (i.e., WRF forecasts).
 The S.~Indian Ocean case shows differences that are clearly
associated with evolving synoptic systems at this time.
 Whereas the E.~Tropical Pacific case shows the smallest
differences where the evolution of the atmosphere is less dynamic.

 \figr{FGATerrors} shows the growth of FGAT errors for
surface wind speeds away from analysis times (0 and 6 UTC).
 Two 6-hour time periods were considered (columns 1 and 2) for
each of three cases (rows 1, 2 and 3).
 Note that the difference between $u$, $v$, and $|V|$ and between
different forecast segments for a single synoptic situation are small
compared to differences between the three synoptic situations studied.

\xx {Accommodating FGAT errors in the VAM\secl{Jnew}}

  \renewcommand{\vardef}[3]{$ {#1} $ is the {#2}}
  \newcommand{\Obs}{Observation}
  \newcommand{\obs}{observation}

According to \cite{HofLH+03} the \obs\ function is given by the sum of
squared differences between simulated (\ie, calculated from the
analysis) and data values normalized by the estimated \obs\ (and
representativeness) error(s).
 For example, for conventional data,
 \eql{Jconv}{ \Jm{CONV} = \sum \frac{(u_a - u_o)^2}{s_u^2} +
 \frac{(v_a - v_o)^2}{s_u^2} . }
 Here
 \vardef{(u_a, v_a)}{analyzed wind interpolated to the \obs\
locations,}{m s^{-1}}
 \vardef{(u_o, v_o)}{observed wind, and}{m s^{-1}}
 \vardef{s_u}{wind component error standard deviation.}{m s^{-1}}

In this formulation $s_u$ implicitly accounts for observation error,
representativeness errors, and interpolation errors.
When using FGAT, the FGAT time interpolation errors vary significantly
from observation to observation, so off time observations should be
given lower weights by inflating the observation error standard
deviation.
Assuming the FGAT time interpolation errors are uncorrelated with the
observational errors already included in $s_u$ we can account for
these errors simply by replacing $s_u^2$ in \eqr{Jconv} with $s_u^2 +
s_{FGAT}^2$. 

In the VAM $s_u$ is taken to be a constant \mks{1}{m s^{-1}} and
knowledge of the \obs\ error standard deviation is absorbed into
\glm{CONV}.
If we consider just one term from \eqr{Jconv}, the \computer{FORTRAN}
code looks like
\eql{Jold}{ \Jm{OLD} = \glm{CODE} (u_a - u_o)^2 ,}
where $\lambda = s_u^2 \glm{CODE}$.
Therefore we will modify each similar term to be
\eql{Jnew1}{ \Jm{NEW} = \lambda \frac{(u_a - u_o)^2}{s_u^2 +
s_{FGAT}^2} = \glm{CODE} \frac{(u_a - u_o)^2}{1+(s_{FGAT}/s_u)^2} ,}
or
\eql{Jnew}{ \Jm{NEW} = \glm{CODE} (u_a - u_o)^2/V_{\gamma},}
where $V_{\gamma}$ is the variance inflation factor.

It is clear from the results of \secr{errors} that rms errors of ocean
surface wind speed and components have a nearly parabolic shape when
plotted as a function of time between $t_0$ and $t_1$.
Therefore we can model the FGAT time interpolation error standard
deviation as
 \eql{sfgat}{s_{FGAT} = s_3\left\{1 - \left(1 - \frac{|t|}{(\delta t/2)}\right)^2\right\} . }
Here, as before, $t$ is time and $\delta t$ is the time between
analysis times.
Also $s_3$ is the maximum value of the FGAT error that occurs at an
offset of 3 hours.
Based on \eqr{sfgat} we then have
\eql{vinf}{V_{\gamma} = 1 + \left[\frac{s_3}{s_u}
\left\{1 - \left(1 - \frac{|t|}{(\delta t/2)}\right)^2\right\}\right] .}

Since $s_3$ varies quite a bit from synoptic situation to synoptic
situation and it would be difficult to quantify this variability and
since this variability dwarfs other variabilities, we will henceforth
approximate \ms{s_3=1}.
It remains to estimate $s_u$.
Typical values of $s_u$ for wind components are 1 to \ms{2} and for
wind speed from 0.7 to \ms{1.5}.
Since we are using large $\lambda$ values to closely fit the data we
are essentially assuming that $s_u$ values are small and we will
choose to use values at the low end of the quoted ranges in accounting
for FGAT interpolation errors.

\bibliography{ams-abbrev,current,nwp,question}
\bibliographystyle{ametsoc}

\newcounter{fig}

\newcommand{\fitem}[3]{\item\label{fig:#1}\begin{minipage}[t]{6in}
 {\slshape {#2}} %%%  {\bf[#1]}
 \\ \medskip \\ \centerline{{\sffamily\upshape\bfseries {#3}}}\end{minipage}}

\begin{list}{\bfseries\upshape Fig.~\arabic{fig}:}
 {\usecounter{fig}
  \setlength{\labelwidth}{0.25in}
  \setlength{\leftmargin}{0cm}
  \setlength{\labelsep}{0.25in}
  \setlength{\parsep}{0.5cm plus0.2cm minus0.1cm}
  \setlength{\itemsep}{0.5cm plus0.2cm}}

 \fitem{wind-diffs}
 {WRF surface winds (left column) and 3-hr (Interpolated-Truth) surface
wind speed differences (right column) for three different cases.}
 {\setlength{\tabcolsep}{1pt}
  \begin{tabular}{|c|c|c|} \hline
  \ylab{\makebox[10mm]{\rule[-4mm]{0mm}{10mm}}} &
  \tbox{Surface Winds} & \tbox{FGAT-Truth} \\ \hline
  \ylab{Typhoon Meranti} &
  \tbox{\includegraphics[bb=52 142 558 648, clip=true, scale=0.3]
         {Meranti2004080800-back.eps}} &
  \tbox{\includegraphics[bb=25 192 597 727, clip=true, scale=0.3, angle=-90]
         {meranti+3hr.eps}}
         \\ \hline \hline
  \ylab{S. Indian Ocean} &
  \tbox{\includegraphics[bb=52 142 558 648, clip=true, scale=0.3]
         {SIndian2004110100-back.eps}} &
  \tbox{\includegraphics[bb=25 192 597 727, clip=true, scale=0.3, angle=-90]
         {s1101+3hr.eps}}
         \\ \hline \hline
  \ylab{E. Tropical Pacific} &
  \tbox{\includegraphics[bb=52 142 558 648, clip=true, scale=0.3]
         {TropPac2005030700-back.eps}} &
  \tbox{\includegraphics[bb=25 192 597 727, clip=true, scale=0.3, angle=-90]
         {t0700+3hr.eps}} \\ \hline 
  \end{tabular}}

 \fitem{FGATerrors}
 {FGAT error versus time for two time periods (columns 1 and 2) for
each of three cases (three rows).}
 {\setlength{\tabcolsep}{1pt}
  \begin{tabular}{|c|c|c|} \hline
  \ylab{\makebox[10mm]{\rule[-4mm]{0mm}{10mm}}} &
  \tbox{8 August 2004} & \tbox{9 August 2004} \\
  \ylab{Typhoon Meranti} &
  \tbox{\includegraphics[bb=70 90 730 530, clip=true, scale=0.3]
         {Meranti2004080800.eps}} &
  \tbox{\includegraphics[bb=70 90 730 530, clip=true, scale=0.3]
         {Meranti2004080900.eps}}
         \\ \hline \hline
  \ylab{\makebox[10mm]{\rule[-4mm]{0mm}{10mm}}} &
  \tbox{1 November 2004} & \tbox{2 November 2004} \\
  \ylab{S. Indian Ocean} &
  \tbox{\includegraphics[bb=70 90 730 530, clip=true, scale=0.3]
         {SIndian2004110100.eps}} &
  \tbox{\includegraphics[bb=70 90 730 530, clip=true, scale=0.3]
         {SIndian2004110200.eps}}
         \\ \hline \hline
  \ylab{\makebox[10mm]{\rule[-4mm]{0mm}{10mm}}} &
  \tbox{7 March 2005} & \tbox{8 March 2005} \\
  \ylab{E. Tropical Pacific} &
  \tbox{\includegraphics[bb=70 90 730 530, clip=true, scale=0.3]
         {TropPac2005030700.eps}} &
  \tbox{\includegraphics[bb=70 90 730 530, clip=true, scale=0.3]
         {TropPac2005030800.eps}} \\ \hline 
  \end{tabular}}

\end{list}

\end{document}
