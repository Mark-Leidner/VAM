%x Emacs: -*- mode:outline; outline-regexp:"[%\\\\]x+ " -*-
%  RCS: $Id: fgat-method.tex,v 1.1 2005/02/15 17:24:47 rnh Exp $
%  Copy of RCS: Id: progress-Aug04.tex,v 1.4 2004/09/29 15:18:05 rnh Exp
%  p1219 progress report as of the end of August 2004.

\documentclass[12pt,titlepage]{article}
\usepackage{times}
\input{format}
\input{abbrev}

\newcommand{\Chapter}[1]{document {\tt #1}}
\newcommand{\Appendix}[1]{document {\tt #1}}
\newcommand{\startAppendixes}{\appendix}
\newcommand{\tabhead}[1]{{\bf #1}}

\newcommand{\authors}{Ross N. Hoffman and S. Mark Leidner}
\newcommand{\RCSid}{$Id: fgat-method.tex,v 1.1 2005/02/15 17:24:47 rnh Exp $}
\newcommand{\RCSrevision}{$Revision: 1.1 $}
\newcommand{\support}{\thanks{ 
        Supported by NASA contract NAS5-32953.
        AER, Inc. intends to retain % (has retained)
        patent rights to certain aspects of the algorithms described herein
        under FAR 52.227-11.}}
\newcommand{\docinfo}{\thanks{ 
        To be submitted to Goddard Space Flight Center (NASA/GSFC),
        Greenbelt, MD 20771.}}
\newcommand{\rightsinfo}[1]{\thanks{ 
        Copyright \copyright\ #1.  
        Work in Progress.  All Rights Reserved.}}
\newcommand{\aerinfo}{\thanks{
        Atmospheric and Environmental Research (AER), Inc.
        131 Hartwell Avenue, Lexington, MA 02421-3126.
        Phone: (781)-761-2288   FAX: (781)-761-2299
        Net: \computer{http://www.aer.com/}.}}
\newcommand{\revinfo}[1]{\thanks{
        AER document version control: \computer{P1219, #1}.
        Formatting version \computer{$Id: fgat-method.tex,v 1.1 2005/02/15 17:24:47 rnh Exp $}.}}
\newcommand{\xtitle}[4]{
        \title{Support for the Development of Cross-Calibrated, \\
               Multi-Platform Ocean Surface Wind Products
% For delivery modify \docinfo and eliminate \rightsinfo
        #1$\!\!$\docinfo} 
        \author{#2\rightsinfo{#2} \\
        Atmospheric and Environmental Research, Inc.\aerinfo }
        \date{#3\revinfo{#4} \\~\\ \today}
        \maketitle
        \markboth {AER, Inc.\ NASA FGAT for the VAM}
        {#1 (#3)}
        \begin{titlepage} \mbox{~} \end{titlepage}
}

\newcommand{\x}[1]{\xtitle{#1}{\authors}{\RCSrevision}{\RCSid}}

\begin{document}

\x {}

\xx {Introduction}

 This is a progress report on AER's support of
the generation of cross-calibrated, multi-platform ocean surface wind
products at NASA Goddard.
 The project at NASA Goddard aims to generate high-quality,
high-resolution, vector winds over the world's oceans between the
years 1987 and 2007 using {\em in situ} and remotely sensed wind
observations.
This report summarizes work supported by AER project number 1219.

 AER is helping specifically by enhancing the capability and
optimizing the use of a two-dimensional data assimilation application
for surface winds called the variational analysis method (VAM).
 The VAM is at the center of the project's analysis procedures for
combining observations of the wind over the world's oceans.

  In this report, the results of software development and testing are
reported in three areas:

  \begin{itemize}
    \item Data assimilation using the First Guess at the Appropriate
    Time (FGAT) (\secr{FGAT})

    \item Data assimilation of SSM/I winds (\secr{SSMI})

    \item Date/time consistency checks (\secr{date-time})
  \end{itemize}

\xx {Data assimilation using the First Guess at the Appropriate Time (FGAT) \secl{FGAT}}

The First Guess at the Appropriate Time (FGAT) procedure was designed
and coded.
The design is detailed in \secr{design}.
Then cases were selected in collaboration with Joe Ardizzone for
\qscat\ in the North Atlantic during October 2003.
Some of these cases were used to test the FGAT code.
We made a subjective and objective assessment of the impact of FGAT
for selected cases \secr{assess}.
The initial implementation was reworked to better integrate the FGAT
preparation with the VAM.
This should make the application of FGAT simpler and more efficient.
The correctness of the new implementation (called
\computer{FGAT\_ADD}) was established by regression testing (\ie, by
repeating previous example cases).
The new version of the VAM that includes the FGAT software and example
test cases were delivered to Joe Ardizzone at NASA Goddard.

\xxx {FGAT design\secl{design}}

FGAT can be used to account for the first order effects of the
asynopticity of the ocean surface wind observations.
For a basic FGAT procedure one uses the time interpolated first guess
to define the observation increments, but assumes the analysis
increments are constant with respect to time.
Another approach is to interpolate in time through the analysis (\ie,
the analysis being determined by the VAM at the synoptic time) at the
center time and through the fixed earlier and later estimates of the
wind field.
As described below both cases can be written as
 \eql{fgat}{V_a(t) = \alpha V_a(t_1) + V_{\delta}.}
Here $V = (u,v)$ is the surface wind, $t$ is time, and $t_1$ is the
synoptic time of the analysis.
Subscripts $a$, $b$, and $\delta$ denote the analysis, the
background, and the FGAT increments, respectively.
Since $V_a(t)$ is only needed to simulate observations \eqr{fgat}
needs only be evaluated at the observation locations.
Therefore we have added $\alpha$, $u_{\delta}$, and $v_{\delta}$ to
the VAM observation data structure.
Preprocessors should  set  $\alpha=1$, $u_{\delta}=0$, and
$v_{\delta}=0$ so that there is no time interpolation.
\computer{FGAT\_ADD} is designed to calculate $\alpha$, $u_{\delta}$, and
$v_{\delta}$ for a variety of options governing the time
interpolation.

In what follows we determine $\alpha$ and $V_{\delta}$ for linear or
quadratic time interpolation, and for FGAT or analysis time
interpolation.
Three possible modes of operation are then described.

\xxxx {Three point time interpolation}

Suppose we are given data values $(x_0, x_1, x_2)$ at three times
$(t_0, t_1, t_2)$.
The interpolated value is linear in the data, so we may write
 \eql{three}{x(t) = c_0(t) x_0 + c_1(t) x_1 + c_2(t) x_2 .}
In the linear case one of $(c_0, c_2)$ will be zero and the
other $c_k$ will vary linearly with $t$.
In the quadratic case the $c_k$ are quadratic functions of time.

To determine the $c_k$ we use the Lagrange form of the collocating
polynomial of degree $n$ for data points  $x_0, x_1, \ldots x_n$:
 \eql{Lagrange}{x(t) = \sum_{k=0}^n x_k
     \left[\prod_{i \ne k} (t-t_i) \right]
     \left[\prod_{i \ne k} (t_k-t_i) \right]^{-1}
   = \sum_{k=0}^n c_k(t) x_k .}
Note that each $c_k$ is a degree $n$ polynomial in $t$ and that
$c_k(t_i) = \delta_{ik}$ thereby ensuring collocation.

For our problem $t_2 - t_1 = t_1 - t_0 = \delta t$.
Then our results for $c_k$ are most concise if we center and normalize
time according to 
 \eql{time}{\Delta = (t - t_1)/\delta t .}
There are three cases to consider.
For linear interpolation between $t_0$ and $t_1$ we have $(c_0 =
-\Delta, c_1 = \Delta + 1, c_2 = 0)$.
For linear interpolation between $t_1$ and $t_2$ we have $(c_0 = 0, c_1
= 1 - \Delta, c_2 = \Delta)$.
For quadratic interpolation between $t_0$ and $t_2$ we have $(c_0 =
\Delta(\Delta - 1)/2, c_1 = -(\Delta + 1)(\Delta - 1), c_2 = (\Delta +
1)\Delta/2)$.
Since $\sum c_k = 1$ we can always calculate $c_1$ as $1-(c_0 + c_2)$.

\xxxx {FGAT time interpolation}

Since the analysis increment is fixed with respect to time we may
write the analysis at any time as the sum of the background at that
time plus the analysis increment:
 \eql{fgat1}{V_a(t) = V_b(t) + [V_a(t_1) - V_b(t_1) .}
In terms of the background values at the three time levels we have
 \eql{fgat2}{V_a(t) = V_a(t_1) + c_0(t) V_b(t_0) + (c_1(t) - 1)
     V_b(t_1) +  c_2(t) V_b(t_2) .}
Comparing this result with \eqr{fgat} we see that $\alpha = 1$ and
$V_{\delta} = c_0(t) V_b(t_0) + (c_1(t) - 1) V_b(t_1) + c_2(t)
V_b(t_2)$.

\xxxx {Analysis time interpolation}

For analysis time interpolation we simply have:
 \eql{anal}{V_a(t) = c_0(t) V_b(t_0) + c_1(t) V_a(t_1) +  c_2(t) V_b(t_2) .}
Comparing this result with \eqr{fgat} we see that $\alpha = c_1(t)$
and $V_{\delta} = c_0(t) V_b(t_0) + c_2(t) V_b(t_2)$.

\xxxx {Modes of operation}

Usually the analysis is applied sequentially.
This allows us to consider three modes of operation, different only
in how $V_b(t_0)$ and $V_b(t_2)$ are defined.
In all three modes $V_b(t_1)$ should be defined as the \apriori\
background.
In the \emph{analysis mode} $V_b(t_0)$ and $V_b(t_2)$ are both the
\apriori\ background.
In the \emph{filter mode} $V_b(t_0)$ is the result of the analysis
determined at the previous synoptic time.
In the \emph{smoother mode}, after a filter mode sequence, we filter
again, but backwards in time so that $V_b(t_0)$ is the result of the
forward filter and $V_b(t_2)$ is the result of the backwards filter.

\xxx {FGAT assessment\secl{assess}}

The VAM using FGAT was tested on eight cases in the North Atlantic using
\qscat\ data.  For each case, a small data window, generally about
\degrees{12} of longitude by \degrees{8} of latitude, was chosen to
highlight a flow feature in the surface winds.

We compare VAM-selected ambiguities (\ie, those closest to
the VAM analysis) to JPL-selected ambiguities.  JPL uses
time-interpolated surface wind analyses to seed their ambiguity
removal.  So, we expect VAM-FGAT (which uses a time-interpolated
background for evaluating misfit) to result in ambiguity selections
which are more similar to JPL-selections than VAM without FGAT.  VAM
without FGAT (we call it "First Guess at Background Time" (FGBT)) may
misplace a feature if the scatterometer data is far in time from the
analysis time.

The first four cases don't show much impact from using FGAT.
The fifth case (00 UTC 21 October 2003) shows a beneficial
impact---ambiguity selection is more like JPL's.
The sixth case (18 UTC 21 October 2003) is clearly a beneficial
impact---ambiguity selection is more like JPL's and a linear feature
is not moved.
In the seventh case (00 UTC 26 October 2003), the scatterometer winds
are at odds with the background and the VAM FGAT analysis opts for
another solution which produces a different set of ambiguity
selections.
In the eight case and last case (06 UTC 26 October 2003), the analysis
was performed using a much bigger data window.
VAM FGAT helps position several features visible in the flow field
which are more in line with the position as found in JPL-selected
ambiguities.

\TBD{Add figures and figure references for these cases}

To document the effects of using FGAT in the VAM, the statistics below
compare the closest scatterometer ambiguity to the background (or
analysis) to the background (or analysis) interpolated to the
scatterometer data location.  
The two successive revs (22659 and 22660) that we analyze are +18 and
+118 minutes, respectively, away from the valid analysis time (0600
UTC 26 October 2003).

In general, FGAT improves the fit of data to the first guess
and the analysis.  But significant improvement (at the 95\% confidence
level) is seen for rev 22660 only, and then only for background wind speed
and background and analysis wind direction.  (Significant improvements
in the tables below are marked with an asterisk.)  Even so, the signal
is clearly in the right direction, especially for rev 22660 whose valid
time is far from the analysis time.  This analysis is presented only
for one case, and it is likely that there are other cases where FGAT
might have a larger impact.

For scalar wind speed the statistics for
(closest.to.background - background) are:
\begin{verbatim}
                   N   mean    mae     sd   rmse   min     max

FGBT rev 22659: 4784  1.680  1.680  1.253  2.096  0.024  15.945
FGAT rev 22659: 4784  1.672  1.672  1.251  2.088  0.035  15.912

FGBT rev 22660: 7424  2.650  2.650  2.719  3.797  0.015  22.055
FGAT rev 22660: 7424  2.503* 2.503  2.722  3.698  0.012  21.941
\end{verbatim}
and for (closest.to.analysis - analysis):
\begin{verbatim}
                   N   mean    mae     sd   rmse   min     max

FGBT rev 22659: 4784  1.415  1.415  1.156  1.827  0.009  15.166
FGAT rev 22659: 4784  1.419  1.419  1.150  1.827  0.029  15.144

FGBT rev 22660: 7424  1.967  1.967  2.408  3.109  0.024  21.602
FGAT rev 22660: 7424  1.957  1.957  2.370  3.074  0.013  21.387
\end{verbatim}
For wind direction the statistics for
ABS(closest.to.background - background) are:
\begin{verbatim}
                   N   mean    mae     sd   rmse   min     max

FGBT rev 22659: 4784  7.903  7.903  8.458 11.575  0.002  95.164
FGAT rev 22659: 4784  8.024  8.024  8.493 11.684  0.010  93.550

FGBT rev 22660: 7424 11.020 11.020 12.041 16.323  0.001 115.100
FGAT rev 22660: 7424 10.531*10.531 11.849 15.853  0.004 119.711
\end{verbatim}
and for ABS(closest.to.analysis - analysis):
\begin{verbatim}
                   N   mean    mae     sd   rmse   min     max

FGBT rev 22659: 4784  6.982  6.982  8.232 10.794  0.000  93.187
FGAT rev 22659: 4784  6.778  6.778  8.268 10.691  0.001  93.607

FGBT rev 22660: 7424  8.650  8.650 10.910 13.923  0.000 104.378
FGAT rev 22660: 7424  8.383* 8.383 10.861 13.720  0.001 123.343
\end{verbatim}

\xx {Data assimilation of SSM/I winds \secl{SSMI}}

In previous work we treated SSM/I wind speed data as a special type
of scatterometer data.
For the current project we have defined a microwave ocean surface wind speed
observation operator appropriate for SSM/I, TMI, AMSR, and other
similar instrument.
This new observations operator can also be applied to any wind speed
data, eg, from WINDSAT.
With these additions the VAM recognizes the following data types:
\computer{'convention', 'ambiguous', 'sigma0', 'ssmi','tmi', 'amsr'}.
SSM/I, AMSR, and TMI will all be handled similarly, but considering
them different data types allows us to weight them differently.
These weights should account for data density and data quality.

For this activity we designed, coded, and tested the new observation
operator.
We were able to reuse code developed for the scatterometer observation
operator in this task thanks to the modular design of the VAM.
We reviewed the new VAM preprocessor modules developed by Joe
Ardizzone to ingest SSM/I wind speed data from RSS.

The VAM including the SSM/I observation operator and test
cases were delivered to Joe Ardizzone at NASA Goddard.

\xx {Date/time consistency checks \secl{date-time}}

We designed, coded, and tested enhancements to the VAM and
preprocessing codes to allow for checking that input data sets
date/times are consistent.

Time stored in the data structure for each datum should be time in
seconds relative to some epoch.
This "zero" time could be the synoptic date/time of the analysis or 00 GMT 1
January 1985 or any other particular time.
In the VAM time for a datum is only used for FGAT and the same units
of time must be used to specify \computer{t2} and \computer{dt} in the
\computer{interp} namelist.

The consistency enhancement adds a call to a routine called
\computer{date\_time\_check} (in \computer{vam\_obs\_mod.f}) before each task.
This routine returns a status indicator.
A positive  status value indicates the number of obs data 
date/times that do not match the grid date/time.
A value of 0 indicates all date/times match.
A value of -1 indicates the grid is not yet defined, -2 that there are
no observation data yet, and -3 that both grid and observation data
are missing.
If a positive value is returned the VAM immediately stops.
If the status is negative and the requested task is CALCulate the VAM
immediately stops.

FGAT checks the date/time pair in the grid against the three date/time 
pairs in namelist \computer{interp}.
\computer{t2} should be consistent with \computer{idate2/itime2},
\computer{t2-dt} with \computer{idate1/itime1} and \computer{t2+dt}
with \computer{idate3/itime3}.
Code to generate \computer{idate1, itime1, idate2,} \ldots, from
\computer{t2} and \computer{dt} could be added by converting the epoch
time (\ie, seconds in the epoch for \computer{dt, t2}, and observation
time) to an date/time pair.

The VAM including the data/time consistency enhancements and test
cases were delivered to Joe Ardizzone at NASA Goddard.

\end{document}
