%x Emacs: -*- mode:outline; outline-regexp:"[%\\\\]x+ " -*-
%  RCS: $Id: fgat-accuracy-estimation.tex,v 1.1 2009/10/03 04:13:01 rnh Exp $
%  Copy of RCS: Id: progress-Aug04.tex,v 1.4 2004/09/29 15:18:05 rnh Exp
%  p1219 progress report as of the end of August 2004.

\documentclass[12pt,titlepage]{article}
\usepackage{times}
\usepackage[dvips]{color}
\input{format}
\input{abbrev}

 % This is needed to bring in ametsoc.sty and use ametsoc.bst
 \usepackage{ametsoc}
 %\usepackage[final]{ametsoc}
 %\usepackage[conf]{ametsoc}

\newcommand{\Chapter}[1]{document {\tt #1}}
\newcommand{\Appendix}[1]{document {\tt #1}}
\newcommand{\startAppendixes}{\appendix}
\newcommand{\tabhead}[1]{{\bf #1}}

\newcommand{\authors}{Ross N. Hoffman}
\newcommand{\RCSid}{$Id: fgat-accuracy-estimation.tex,v 1.1 2009/10/03 04:13:01 rnh Exp $}
\newcommand{\RCSrevision}{$Revision: 1.1 $}
\newcommand{\support}{\thanks{ 
        Supported by SAIC contract 4400125404, a subcontract to 
	NASA contract NAS5-nnnnn.
        AER, Inc. intends to retain % (has retained)
        patent rights to certain aspects of the algorithms described herein
        under FAR 52.227-11.}}
\newcommand{\docinfo}{\thanks{ 
        To be submitted to Goddard Space Flight Center (NASA/GSFC),
        Greenbelt, MD 20771.}}
\newcommand{\rightsinfo}[1]{\thanks{ 
        Copyright \copyright\ #1.  
        Work in Progress.  All Rights Reserved.}}
\newcommand{\aerinfo}{\thanks{
        Atmospheric and Environmental Research (AER), Inc.
        131 Hartwell Avenue, Lexington, MA 02421-3126.
        Phone: (781)-761-2288   FAX: (781)-761-2299
        Net: \computer{http://www.aer.com/}.}}
\newcommand{\revinfo}[1]{\thanks{
        AER document version control: \computer{P1326, #1}.
        Formatting version \computer{$Id: fgat-accuracy-estimation.tex,v 1.1 2009/10/03 04:13:01 rnh Exp $}.}}
\newcommand{\xtitle}[4]{
        \title{Support for the Development of Cross-Calibrated, \\
               Multi-Platform Ocean Surface Wind Products
% For delivery modify \docinfo and eliminate \rightsinfo
        #1$\!\!$\docinfo} 
        \author{#2\rightsinfo{#2} \\
        Atmospheric and Environmental Research, Inc.\aerinfo }
        \date{#3\revinfo{#4} \\~\\ \today}
        \maketitle
        \markboth {AER, Inc.\ NASA FGAT for the VAM}
        {#1 (#3)}
        \begin{titlepage} \mbox{~} \end{titlepage}
}

\newcommand{\x}[1]{\xtitle{#1}{\authors}{\RCSrevision}{\RCSid}}

\begin{document}

\x {}

\xx {Introduction}

 This is a technical report in AER's support of the generation of
cross-calibrated, multi-platform (CCMP) ocean surface wind products at
NASA Goddard.
 The project at NASA Goddard aims to generate high-quality,
high-resolution, vector winds over the world's oceans beginning with
the 1987 launch of the SSM/I F08, and using both {\em in situ} and remotely
sensed wind observations.

In this project AER is helping specifically by enhancing the capability and
optimizing the use of a two-dimensional data assimilation application
for surface winds called the variational analysis method (VAM).
 The VAM is at the center of the project's analysis procedures for
combining observations of the wind over the world's oceans.

This report summarizes an approach to provide uncertainty estimates
for the VAM analyses.
To obtain these estimates we will approximate the VAM as a least
squares weighted average of $n$ independent observations, treating the
background as an observation and ignoring the small amount of
conventional data.
This approach is applicable most directly to estimating the accuracy
of wind speed.
Extensions to direction, wind components, and pseudo-stress components
are discussed.

\xx {Error standard deviation, error variance, and accuracy\secl{Lei81b}}

\citet{Lei81b} defines accuracy, $A$, as the inverse of error variance,
$E$, where $E$ is the square of the standard deviation, $s$. That is,
\eql{A}{A=1/E, \qquad\mbox{and}}
\eql{E}{E=s^2.}

For independent error source the error variances add. That is, if a
measurement is subject to error sources 1, 2, 3, and we can estimate
the associated error variances $E_1, E_2, E_3$, then the total error
variance is given by 
\eql{sumE}{E = E_1 + E_2 + E_3.}

For observations with independent errors, the accuracies add. That is
if we have three observations of the $u$-component wind at a specific
time and location with associated error variances $E_1, E_2, E_3$ then
if these observations are unbiased and combined with optimal weights
(\ie, least squares weights) then the accuracy of the weighted average is
\eql{sumA}{A = A_1 + A_2 + A_3 = E_1^{-1} + E_2^{-1} + E_3^{-1} =
E^{-1}.}

The optimal weights sum to one and the $i$th weight is proportional to
$A_i$. In the case of combining two observations,
\eql{combine2}{X = w_1 X_1 + w_2 X_2,}
we have,
\eql{weight}{w_1=\frac{A_1}{A_1+A_2} \qquad\mbox{and}\qquad
w_2=\frac{A_2}{A_1+A_2}.}

These results for accuracy are strictly true if the observations
are unbiased, independent, and combined with an optimally weighted
averaging process.
Leith extends the analysis for vector quantities with $E$ the error
covariance matrix and $A$ the inverse of $E$.
Initially we will consider only the accuracy of the speed.
But the extension to 2-component vector quanitities might be useful in
treating speed and direction together, or for wind components or for
pseudo-stress components.

\xx {Estimating the VAM accuracies}

In the CCMP, the observations and VAM analyses are both on the same
\degrees{1/4} Wentz grid and the ECMWF backgrounds can be interpolated
to that grid.
Thus we can easily calculate and/or accumulate statistics on the
analysis grid.
Here we ignore the fact that the VAM makes use of neighboring data.
This assumption is valid inside swaths of satellite data which have
spatially correlated observation errors, but would have to be
revisited if we wished to include the conventional data in our
accuracy computation.

\xxx {Accuracy of individual VAM analyses}

We wish to apply \eqr{sumA} at each analysis time and each grid point,
but with \eqr{sumA} extended from 3 pieces of information to $1 + n$
pieces of information, where $n$ is the number of satellite
observations used at the particular analysis time and grid point.

This estimation will be done for wind speed initially, but might be
extended to wind components, speed,
direction, and pseudo-stress components separately or pairwise.
We need accuracy estimates for speed and eventually the other wind
quantities for each instrument used and for the background.
Each accuracy estimate, except for the background, should be modified
to account for the time difference from the analysis time following,
the FGAT procedure.

We claim that this processs will underestimate the VAM analysis
accuracy because the VAM includes other information in the form of
constraints, conventional data, and neighboring satellite observations.

\xxx {Accuracy of time averages of VAM analyses}

Accuracies continue to add if we form optimally weighted averages of
the VAM analyses.
However we will generally form simple averages of the VAM analyses
that correspond to time averages.
If the errors are independent, and we average over $N$ VAM analyses
then applying \eqr{sumE} and dividing by $N$, we have that the error
of the time average is given by
\eql{time-average}{E=\frac{1}{N}\sum_{i=1}^{N}E_i =
\frac{1}{N}\sum_{i=1}^{N} \frac{1}{A_i} = A^{-1}.}

\xx {Error standard deviation estimates\secl{sdev}}

To estimate the VAM analysis accuracy we need estimates of the
estimated error standard deviations of speed for each observation.
These estimates should be consistent with the weights used in the
VAM, including the FGAT modification.
We have assumed that measurement errors are small, with 
\ms{s_u=s_v=1} and that \ms{s_w=0.7} where the
subscripts $u, v, w$ are for the two wind components and wind speed.
NB: because of the subjectivity in these assignments and the tuning of
the $\lambda$-weights I think we must tune the values of $s_w$ for the
background.
We might try first with applying/tuning this approach to a recent year
in which we withhold both conventional and scatterometer data.

The FGAT modification is provided in the Appendix (\secr{Jnew}).
The FGAT error variance, given by \eqr{sfgat} with \ms{s_3=1}, is then
added to the measurement error variance to give the observational
error variance.

For direction and other quantities we must recognize that the
microwave radiometers are not supplying independent directional
information.
The VAM does extract some directional information by combining the
speed observations with the smoothing and dynamical constraints, but
we can not treat this here.
When scatterometer data are not available this would imply that the
wind directional accuracy comes only from the ECMWF backgroun.

To extend to 2-component vector quantities---speed and direction, wind components or
pseudo-stress components---will require additional work.
A faithful probability distribution function for the errors of an
instrument observing the wind should allow the evaluation of the
$2\times2$ matrix $E$ for any one of these pairs.

\xx {Consistency verification: Application to collocation statistics}

Underlying the approach presented here to estimating the VAM
accuracies are estimates of the observation and background error
standard deviations.
Comparing actual collocations statistics to estimates of the rms
collocation differences based on the method of \secr{Lei81b} and the data of
\secr{sdev} provides a verification and opportunity to tune the error
standard deviations.
We typically calculate the mean (or bias) and rms difference for
collocations.
In what follows we assume the bias has been removed or is small enough
to ignore.

\xxx {Observation -- Background ($O-B$) differences} 

Whether adding or subtracting two independent error sources, the error
variances add.
To demonstrate this assume the observation and background errors are
independent.
Then we have
\eql{OMB}{E_d = <(X_o - X_b)^2> = <(e_o - e_b)^2> = E_o + E_b.}
Here $E_d$, $E_o$, and $E_b$ are the error variances of the
difference, observation, and background, respectively, and $<X>$ is
the expectation of $X$.
In \eqr{OMB} we first subtract and add the truth ($X_t$) to the
difference between observation and background ($X_o - X_b$), to get
the difference of the errors of the observation the background ($e_o -
e_b$) since $e_o = X_o - X_t$ and $e_b = X_b - X_t$, and then make use
of the fact that these two errors are uncorrelated.
In terms of observation and background accuracies ($A_o$ and $A_b$) we
can rewrite $E_d$ as 
\eql{OMB-Accuracy}{E_d = A_o^{-1} + A_b^{-1} =  A_d^{-1}.}
This formulation would also apply to comparisons of an analysis to a
data set not used in that analysis.

\xxx {Observation -- Analysis ($O-A$) differences} 

To compare the VAM analysis (denoted $X$) to one data source
(denoted $X_1$) used in the analysis we consider the analysis to be
the optimally weighted sum of an analysis that uses all other data
sources (denoted $X_2$) and the one data source collocated with the
VAM analysis.
Note that we do not actually need to calculate $X_2$ or $A_2$.
A special case occurs when $X_1$ is the ECMWF background, in which case the
collocation statistics are for the $B-A$ differences.

Making use of \eqr{combine2}, noting that $w_1 + w_2 = 1$, we have
\eql{OMA}{\begin{array}{rcl}E_d & = & <(X_1 - X)^2> = <(X_1 - [w_1 X_1 + w_2 X_2])^2> \\
              & = & w_2^2 <(X_1 - X_2)^2> = w_2^2 <(e_1 - e_2)^2>
              = w_2^2 (E_1 + E_2).\end{array}}
In \eqr{OMA} we add and subtract the truth as in the derivation of \eqr{OMB} and make
use of the fact that the errors ($e_1$ and $e_2$) of $X_1$ and $X_2$
are independent.
In terms of observation and analysis accuracies ($A_1$ and $A$) we
can rewrite $E_d$ as 
\eql{OMA-Accuracy}{E_d = \frac{A_2^2}{(A_1 + A_2)^2}\left(\frac{1}{A_1} +
\frac{1}{A_2}\right) = A_1^{-1} - A^{-1} = E_1 - E,}
making use of \eqr{weight} and \eqr{sumA} (\ie, $A_2=A-A-1$), and
where in the final step we have rewritten our answer in terms of error variances. 	

\appendix

\xx {Accommodating FGAT errors in the VAM\secl{Jnew}}

  \renewcommand{\vardef}[3]{$ {#1} $ is the {#2}}
  \newcommand{\Obs}{Observation}
  \newcommand{\obs}{observation}

According to \citet{HofLH+03} the \obs\ function is given by the sum of
squared differences between simulated (\ie, calculated from the
analysis) and data values normalized by the estimated \obs\ (and
representativeness) error(s).
 For example, for conventional data,
 \eql{Jconv}{ \Jm{CONV} = \sum \frac{(u_a - u_o)^2}{s_u^2} +
 \frac{(v_a - v_o)^2}{s_u^2} . }
 Here
 \vardef{(u_a, v_a)}{analyzed wind interpolated to the \obs\
locations,}{m s^{-1}}
 \vardef{(u_o, v_o)}{observed wind, and}{m s^{-1}}
 \vardef{s_u}{wind component error standard deviation.}{m s^{-1}}

In this formulation $s_u$ implicitly accounts for observation error,
representativeness errors, and interpolation errors.
When using FGAT, the FGAT time interpolation errors vary significantly
from observation to observation, so off time observations should be
given lower weights by inflating the observation error standard
deviation.
Assuming the FGAT time interpolation errors are uncorrelated with the
observational errors already included in $s_u$ we can account for
these errors simply by replacing $s_u^2$ in \eqr{Jconv} with $s_u^2 +
s_{FGAT}^2$. 

In the VAM $s_u$ is taken to be a constant \mks{1}{m s^{-1}} and
knowledge of the \obs\ error standard deviation is absorbed into
\glm{CONV}.
If we consider just one term from \eqr{Jconv}, the \computer{FORTRAN}
code looks like
\eql{Jold}{ \Jm{OLD} = \glm{CODE} (u_a - u_o)^2 ,}
where $\lambda = s_u^2 \glm{CODE}$.
Therefore we will modify each similar term to be
\eql{Jnew1}{ \Jm{NEW} = \lambda \frac{(u_a - u_o)^2}{s_u^2 +
s_{FGAT}^2} = \glm{CODE} \frac{(u_a - u_o)^2}{1+(s_{FGAT}/s_u)^2} ,}
or
\eql{Jnew}{ \Jm{NEW} = \glm{CODE} (u_a - u_o)^2/V_{\gamma},}
where $V_{\gamma}$ is the variance inflation factor.

It is clear from our earlier results that rms errors of ocean
surface wind speed and components have a nearly parabolic shape when
plotted as a function of time between $t_0$ and $t_1$.
Therefore we can model the FGAT time interpolation error standard
deviation as
 \eql{sfgat}{s_{FGAT} = s_3\left\{1 - \left(1 - \frac{|t|}{(\delta t/2)}\right)^2\right\} . }
Here, as before, $t$ is time and $\delta t$ is the time between
analysis times.
Also $s_3$ is the maximum value of the FGAT error that occurs at an
offset of 3 hours.
Based on \eqr{sfgat} we then have
\eql{vinf}{V_{\gamma} = 1 + \left[\frac{s_3}{s_u}
\left\{1 - \left(1 - \frac{|t|}{(\delta t/2)}\right)^2\right\}\right] .}

Since $s_3$ varies quite a bit from synoptic situation to synoptic
situation and it would be difficult to quantify this variability and
since this variability dwarfs other variabilities, we will henceforth
approximate \ms{s_3=1}.
It remains to estimate $s_u$.
Typical values of $s_u$ for wind components are 1 to \ms{2} and for
wind speed from 0.7 to \ms{1.5}.
Since we are using large $\lambda$ values to closely fit the data we
are essentially assuming that $s_u$ values are small and we will
choose to use values at the low end of the quoted ranges in accounting
for FGAT interpolation errors.

\bibliography{ams-abbrev,current,nwp,question}
\bibliographystyle{ametsoc}

\end{document}
